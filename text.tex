\documentclass[a4paper,14pt]{article}
\linespread{1.5}

\usepackage{cmap}					% поиск в PDF
\usepackage[T2A]{fontenc}			% кодировка
\usepackage[utf8]{inputenc}			% кодировка исходного текста
\usepackage[english,russian]{babel}	% локализация и переносы

\author{Anastasia Nichiporchuk}
\title{Drug Discovery via Graph Neural Networks}
\date{\today}

\begin{document} % Конец преамбулы, начало текста.

\maketitle

\section*{Abstract}

\section{Introduction}\label{1}

The need to treat illness and maintain health has accompanied humans throughout their existence as a species. The development of the pharmaceutical industry has improved the quality of life, increased its duration, and made it possible to cope with diseases that were previously considered incurable. The development of vaccines for preventive protection against disease, both for the individual and for society as a whole, as demonstrated by the recent COVID-19 pandemic, is also crucial. 

Modern drug development is inextricably linked to the use of various technologies. Pharmaceutical companies compete for primacy in the development of drugs for diseases that place the greatest burden on the health care system. Before the discovery of penicillin, these were mostly diseases of an infectious nature. Nowadays, cardiovascular diseases, cancer and neurodegenerative diseases account for the largest percentage.

Because time is of the essence in drug development, pharmaceutical companies use a variety of cutting-edge research methods. In addition to laboratory research, computer simulations of different processes, mathematical modeling, and other methods that can reduce the amount of laboratory research are actively used. 


This is due to the fact that conducting any laboratory research requires time and financial expenses. Also, you need highly qualified specialists who are able to conduct this research. For example, a substance can be tested in the laboratory for a fairly small number of properties, so you need to try empirically to exclude those tests that are known to give a bad result.

With the development of computing power, as well as biotechnology and cheminformatics, it has become possible to use neural networks as such empirical methods. A neural network trained for a specific task can predict certain data and minimize unsuccessful laboratory tests.

\section{Solubility}\label{2}
Solubility is the maximum amount of solute that can be dissolved in a known amount of solvent at a particular temperature.

Factors affecting solubility:
\begin {itemize}
\item {temperature. Solubility depends significantly on temperature and can be changed by increasing or lowering the temperature. As a rule, the range from 20°C to 100 °C is considered suitable for dissolution in water.}

\item{pressure. Pressure has practically no effect on the solubility of solids, but for gaseous substances, solubility is directly proportional to the pressure of the gas over the liquid.}

\item{the nature of the solvent and the solute. Water belongs to polar solvents and, accordingly, dissolves polar substances better.}
\end {itemize}

Aqueous solubility, denoted by S, or its logarithm value log Is very important molecular property. Identification of
molecules with undesirable solubility in water at early stages is of great importance in drug development and in other related fields of pharmacy, since solubility affects the processes of absorption, distribution, metabolism and elimination (ADME).

A number of existing QSPR models developed to predict solubility show that solubility is related to experimental indicators such as melting point and separation coefficient.
However, the data obtained experimentally often contain measurement errors. This increases the difficulty of predicting solubility.

The most using formula of solubility is:
\begin {equation}
 logS =  0.5 - 0.01(MP - 25) - logP,
\end{equation}
where MP is melting point and logP is the log of the octanol-water partition coefficient.

Both partition coefficient and aqueous solubility reveal how
absolute dissolves in a solvent.

Solubility of molecules in water is an important property in terms of drug development. Substances with good solubility can be incorporated into drugs "as is," while substances with low solubility require the addition of various impurities to increase the final solubility. 

The most popular forms of drugs are tablets and solutions for injection. It is not uncommon for laboratory tests to find a substance that has the desired effect (capable of acting as a medicine for a specific disease). Such a substance needs to be submitted for clinical trials already in the dosage form in which it is supposed to be taken by patients. At this stage there may be a hitch, precisely because the active ingredient is poorly soluble in water, and it is not technologically easy to make a tablet from it.

Solubility can also affect the absorption of the active ingredient and its stability while in the dosage form. Thus, it can be determined that the better the solubility of the substance in water, the easier and cheaper it will be to use to make the drug.

\section{Graph Neural Networks}\label{3}

Graph neural networks are a class of neural networks that allow you to apply deep learning techniques to a variety of data that can be represented as a graph. Graph data has no regular structure, like pictures (it is known exactly how many neighbors a pixel has) or texts (there is a sequence of words in a sentence). Therefore, classical neural networks are not well suited to this kind of data. At the same time, the variety of such data in the real world is very large - as an example, we can consider social networks, routes, and molecular structures. 

There are several different types of graph problems:
\begin{enumerate}
  \item node-level prediction
  \item edge-level prediction
  \item graph-level prediction
  \item community detection (the problem of searching for clusters in a graph).
\end{enumerate}

The first type can include various tasks of prediction and classification of graph vertices, for example, prediction of the topic of an article on the basis of citations. 

The second type includes tasks related to a pair of vertices or edges, the so-called "link prediction" tasks. As an example of such a problem, we can consider recommendation systems that predict whether a user is suitable for a service offered. 

The third type, graph prediction, involves problems where a graph is treated as an object and the output is an estimate of its property. These can be both classification and regression problems. Such problems are often used when considering molecules to predict their properties.

Clustering tasks belong to unsupervised learning and can be used, for example, to distinguish groups of familiar users in social networks.

Below we will consider in more detail in which areas graph neural networks are used.

\subsection{Graph neural networks in different tasks}\label{4}

\subsubsection*{Computer vision}

In computer vision classical neural network models are actively used, in particular CNN. However, if we consider an image as a graph, in which each pixel is a vertex of the graph, and the neighborhood with other pixels determines the presence of edges between the corresponding vertices, it becomes possible to apply GNN for appropriate tasks related to images.

In addition, graphs are actively used for more complex tasks involving relations between image objects.
One such task is scene graph generation, in which the goal of the model is to parse an image into a semantic graph consisting of objects and their semantic relations. From the image data, the model detects and recognizes objects and predicts semantic relationships between pairs of objects.

Another important task is to match objects in scenes. It makes it possible to compile three-dimensional models from the available video footage, which are used for the modeling and mapping (SLAM). If certain hardware conditions are met, such models can also work in real time.

\subsubsection*{Natural language processing}
In natural language processing it is possible to represent a sequence of words as a simple graph. Such a format seems quite intuitive, but not much in demand. The representation of natural language in the form of graphs, such as dependency graphs, constituency graphs, AMR graphs and knowledge graphs, as well as text graphs containing multiple hierarchies of elements, i.e. document, sentence and word.

// TODO: дописать про комбинацию NLP+GNN

\subsubsection*{Recommendation systems}
Recommendation systems are designed to analyze user interests and predict what will be interesting to a particular user at the moment.

Recommendation systems are divided into two fundamentally different types - off-line models (detection of global patterns, personalized model for a specific object) and online models (fast response, detection of current actual trends). 

Graphs are actively used for offline models. are used by large companies in their products (Alibaba, Uber, Pinterest) to prepare personalized offers for users based on their actions and reactions. In this case, the user and the product can be represented as vertices of the graph, and their interaction as an edge of the graph.

\subsubsection*{Natural science}

Chemistry, biology, and other natural sciences have been and remain one of the main driving forces in the development of GNN. The reason for this is that these sciences pose problems closely related to data in the form of graphs - molecules and materials. Also, these problems often require significant computational power when solved by classical methods, so it is appropriate to use machine learning methods for them. In general, when comparing classical methods of machine learning and GNN, the latter show better results.

\section{Recent work}
In this paper, the problems of predicting the solubility of a substance in water based on the structure of a molecule are considered. As described above, solubility is an important property for the production of drugs, which can be calculated using experimental (and therefore not very accurate) data. Therefore, it is advisable to use a neural network to predict this property.

Since the molecule is essentially a graph, it is possible to use a graph neural network in the problem, preserving the original structure of the input data.

// TODO: дописать тут про генеративную часть задачи
\end{document} % Конец текста.

// https://users.math.msu.edu/users/weig/paper/p223.pdf
// https://habr.com/ru/company/vk/blog/557280/
// https://github.com/Dyakonov/DL/blob/master/ESSE_2021/E103_GNN.pdf
// https://drive.google.com/file/d/1Oxo4WHLjDBCNoqJX6uxf9KzdF4f1d9tO/view
// https://mlcourse.at.ispras.ru/wp-content/uploads/2020/11/%D0%9D%D0%B5%D0%B9%D1%80%D0%BE%D0%BD%D0%BD%D1%8B%D0%B5_%D1%81%D0%B5%D1%82%D0%B8_%D0%B4%D0%BB%D1%8F_%D0%B0%D0%BD%D0%B0%D0%BB%D0%B8%D0%B7%D0%B0_%D0%B3%D1%80%D0%B0%D1%84%D0%BE%D0%B2.pdf
// https://www.researchgate.net/publication/351495116_Fusion_of_text_and_graph_information_for_machine_learning_problems_on_networks
// https://www.nature.com/articles/s41597-019-0151-1
// https://www.organic-chemistry.org/prog/peo/logS.html - картинка с распределением
